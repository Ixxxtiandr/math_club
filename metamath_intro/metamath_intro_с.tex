\documentclass[a4paper, 12pt]{article}  % добавить leqno в [] для нумерации слева

%%% Работа с русским языком
\usepackage{cmap}
\usepackage[utf8]{inputenc}
\usepackage[english, russian]{babel}
\usepackage{pscyr}

%%% Дополнительная работа с математикой
\usepackage{amsmath,amsfonts,amssymb,amsthm,mathtools} % AMS
\usepackage{icomma} % "Умная" запятая

%% Номера формул
\mathtoolsset{showonlyrefs=true} % Показывать номера только у тех формул, на которые есть \eqref{} в тексте.

%% Шрифты
\usepackage{euscript} % Шрифт Евклид
\usepackage{mathrsfs} % Красивый матшрифт

%% Первый абзац после секшиона
\usepackage{indentfirst}

%% Свои команды
\DeclareMathOperator{\pphi}{\mathop{\varphi}}
\DeclareMathOperator{\eps}{\mathop{\varepsilon}}
\DeclareMathOperator{\conj}{\mathbb{\&}}

%% Перенос знаков в формулах (по Львовскому)
\newcommand*{\hm}[1]{#1\nobreak\discretionary{}
	{\hbox{$\mathsurround=0pt #1$}}{}}

%% Поля
\usepackage[left=1cm, right=1.4cm, top=1cm, bottom=1.3cm]{geometry}

%% Графика
\usepackage{tikz}

%% Переименовать список литературы
\addto\captionsrussian{\def\refname{Литература}}

%%% Заголовок
\author{Мат.клуб ``Тифаретник'' по С. Клини}
\title{Введение в метаматематику на троечку}
\date{\today}

%%% Эпиграф
\usepackage{epigraph}
\setlength\epigraphwidth{9cm}

%%% Работа с картинками
\usepackage{graphicx}  % Для вставки рисунков
\graphicspath{{images/}{images2/}}  % папки с картинками
\setlength\fboxsep{3pt} % Отступ рамки \fbox{} от рисунка
\setlength\fboxrule{1pt} % Толщина линий рамки \fbox{}
\usepackage{wrapfig} % Обтекание рисунков и таблиц текстом

%%% Добавляем код. Окружение lstlisting
\usepackage{listings}
\usepackage{color}
\definecolor{mygreen}{rgb}{0,0.6,0}
\definecolor{bggray}{RGB}{244,244,244}
\definecolor{mygray}{rgb}{0.5,0.5,0.5}
\definecolor{mymauve}{rgb}{0.58,0,0.82}
\lstset
{
	backgroundcolor=\color{bggray},   % choose the background color; you must add \usepackage{color} or \usepackage{xcolor}; should come as last argument
	basicstyle=\footnotesize,        % the size of the fonts that are used for the code
	breakatwhitespace=false,         % sets if automatic breaks should only happen at whitespace
	breaklines=true,                 % sets automatic line breaking
	captionpos=b,                    % sets the caption-position to bottom
	commentstyle=\color{mygreen},    % comment style
	deletekeywords={...},            % if you want to delete keywords from the given language
	escapeinside={\%*}{*)},          % if you want to add LaTeX within your code
	extendedchars=true,              % lets you use non-ASCII characters; for 8-bits encodings only, does not work with UTF-8
	%	firstnumber=1000,                % start line enumeration with line 1000
	%	frame=single,	                   % adds a frame around the code
	keepspaces=true,                 % keeps spaces in text, useful for keeping indentation of code (possibly needs columns=flexible)
	keywordstyle=\color{blue},       % keyword style
	language=c,                 	 % the language of the code
	morekeywords={*,...},            % if you want to add more keywords to the set
	numbers=none,                    % where to put the line-numbers; possible values are (none, left, right)
	numbersep=5pt,                   % how far the line-numbers are from the code
	numberstyle=\tiny\color{mygray}, % the style that is used for the line-numbers
	rulecolor=\color{black},         % if not set, the frame-color may be changed on line-breaks within not-black text (e.g. comments (green here))
	showspaces=false,                % show spaces everywhere adding particular underscores; it overrides 'showstringspaces'
	showstringspaces=false,          % underline spaces within strings only
	showtabs=false,                  % show tabs within strings adding particular underscores
	stepnumber=2,                    % the step between two line-numbers. If it's 1, each line will be numbered
	stringstyle=\color{mymauve},     % string literal style
	tabsize=3,	                     % sets default tabsize to 2 spaces
	title=\lstname                   % show the filename of files included with \lstinputlisting; also try caption instead of title	
}

%%% Ссылки
\usepackage{hyperref}

%%% Окружения
\theoremstyle{definition}
\newtheorem{theorem}{Теорема}
\newtheorem*{definition}{Определение}
\newtheorem*{pr}{Доказательство}

%%% Цифры в кружке
\newcommand*\circled[1]{\tikz[baseline=(char.base)]{
		\node[shape=circle,draw,inner sep=2pt] (char) {#1};}}

%%% Перечисление кастомное
\usepackage{enumitem}

%%% Естественный вывод
\usepackage{bussproofs}

%%% Несколько колонок для перечислений
\usepackage{multicol}

\usepackage{array}

\begin{document}
	\maketitle
	
	\epigraph{
		Формальные ограничители нужны человеку всегда,
			
		Они как огнетушители, а это, бля, не ерунда.
	}{Кровосток -- Снайпер}


	\begin{definition}[\textsection 16,  Формальные символы]
		\leavevmode	
		\begin{itemize}
			\setlength\itemsep{-3pt}
			\item \textit{Логические символы}: 
				$\supset$(влечёт), $\conj$(и), $\vee$(или),
				$\neg$(не), $\forall$(для всех), $\exists$(существует)
			\item \textit{Символы предикатов}: $=$(равняется)
			\item \textit{Символы функций}: 
				$+$(сложить с), $\cdot$(умножить на), $\backprime$(следующий за)
			\item \textit{Индивидуальные символы}: $0$(нуль)
			\item \textit{Переменные}: $a$, $b$, $c$, \dots
			\item \textit{Скобки}: $($, $)$
		\end{itemize}
	\end{definition}
	
	\begin{definition}[\textsection 17]
		\leavevmode			
		\begin{enumerate}
			\setlength\itemsep{-3pt}	
			\item $0$ есть \textit{терм}
			\item Каждая переменная есть \textit{терм}
			\item Если $s$ и $t$ --- \textit{термы}, то
			\begin{multicols}{4}
				\begin{enumerate}
				\item $(s)+(t)$          --- \textit{терм}
				\item $(s) \cdot (t)$    --- \textit{терм}
				\item $(s)^{\backprime}$ --- \textit{терм}
				\end{enumerate}
			\end{multicols}
			\item Никаких других \textit{термов}, кроме определённых согласно 1--3, нет.
		\end{enumerate}
	\end{definition}

	\begin{definition}[\textsection 17]
		\leavevmode				
		\begin{enumerate}
			\setlength\itemsep{-3pt}	
			\item Если $s$ и $t$ --- \textit{термы}, то $(s)=(t)$ --- \textit{формула}
			\item Если $A$ и $B$ --- \textit{формулы}, то
			\begin{multicols}{3}
				\begin{enumerate}
				\item $(A) \supset (B)$ --- \textit{формула}
				\item $(A) \conj\: (B)$ --- \textit{формула}
				\item $(A) \vee (B)$    --- \textit{формула}
				\item $\neg (A)$        --- \textit{формула}
				\end{enumerate}
			\end{multicols}
			\item Если $x$ --- переменная, а $A$ --- \textit{формула}, то 
			\begin{multicols}{3}
				\begin{enumerate}
				\item $\forall x (A)$ --- \textit{формула}
				\item $\exists x (A)$ --- \textit{формула}
				\end{enumerate}
			\end{multicols}
			\item Никаких других \textit{формул}, кроме определённых согласно 1--3, нет.
		\end{enumerate}
	\end{definition}
	
	\begin{definition}[\textsection 18]
		Вхождение $x$ в формулу $A$ называется \textit{связанным} (или вхождением в качестве 
		\textit{связанной переменной}), если оно является вхождением в квантор $\forall x$ или $\exists x$
		или в область действия квантора $\forall x$ или $\exists x$; в противном случае вхождение
		называется \textit{свободным}. 	
	\end{definition}

	\begin{definition}[\textsection 18]
		\textit{Подстановка терма $t$ вместо} переменной $x$ в терм или формулу $A$ состоит в одновременной
		 замене	каждого \textit{свободного вхождения} $x$ в $A$ на вхождение $t$.		
	\end{definition}

	\begin{definition}[\textsection 18]
		Будем говорить, что терм $t$ \textit{свободен при свободных вхождениях} переменной $x$ в 
		формулу $A(x)$, если никакое свободное вхождение $x$ в $A(x)$ не входит в область действия 
		какого-нибудь квантора $\forall y$ или $\exists y$, где $y$ --- переменная из $t$
		(т.е. входящая в $t$).  		
	\end{definition}

	\newpage
	
	{
		\centering
		\section*{Постулаты формальной системы (\textsection 19)}
	}
		\subsubsection*{Dramatis personae}
			В постулатах 1--8 $A, B$ и $C$ --- формулы. В 9--13 $x$ --- переменная, $A(x)$ --- формула, 
			$C$ --- формула, не содержащая свободно $x$, а $t$ --- терм, свободный для $x$ в $A(x)$.
		
		\subsection*{Группа А. Постулаты исчисления предикатов}
		
		\subsubsection*{Группа А1. Постулаты исчисления высказываний}
			\begin{multicols}{2}
				\begin{itemize}[label={}]
				\setlength\itemsep{0pt}	
				\item \circled{1a} $A \supset (B \supset A)$
				\item \circled{1b} $(A \supset B) \supset ((A \supset (B \supset C)) \supset (A \supset C))$
				\item \circled{4a} $A \conj B \supset A$
				\item \circled{4b} $A \conj B \supset B$
				\item \circled{5a} $A \supset A \vee B$
				\item \circled{5b} $B \supset A \vee B$
				\item \circled{2} 
					\AxiomC{$A, A \supset B $}
					\UnaryInfC{$B$}
					\DisplayProof
				\item \circled{3} $A \supset (B \supset A \conj B)$
				\item \circled{6} $(A \supset C) \supset ((B \supset C) \supset (A \vee B \supset C))$
				\item \circled{7} $(A \supset B) \supset ((A \supset \neg B) \supset \neg A)$
				\item \circled{$8^{\circ}$} $\neg \neg A \supset A$
				\item[\vspace{\fill}]
				\end{itemize}
			\end{multicols}
		
		\subsubsection*{Группа А2. (Дополнительные) Постулаты исчисления предикатов}
		\begin{multicols}{2}
			\begin{itemize}[label={}]
			\item \circled{9}
				\AxiomC{$C \supset A(x)$}
				\UnaryInfC{$C \supset \forall x A(x)$}
				\DisplayProof
			\item \circled{10} $\forall x A(x) \supset A(t)$
			\item \circled{11} $A(t) \supset \exists x A(x)$
			\item \circled{12} 
				\AxiomC{$A(x) \supset C$}
				\UnaryInfC{$\exists x A(x) \supset C$}
				\DisplayProof
			\end{itemize}
		\end{multicols}
		
		\subsection*{Группа B. (Дополнительные) Постулаты арифметики}
			\begin{multicols}{2}
				\begin{itemize}[label={}]
				\setlength\itemsep{0pt}	
				\item \circled{13} $A(0) \conj \forall x (A(x) \supset A(x^{\backprime})) \supset A(x)$
				\item \circled{14} $a^{\backprime} = b^{\backprime} \supset a = b$
				\item \circled{15} $\neg a^{\backprime} = 0$
				\item \circled{16} $a = b \supset (a = c \supset b = c)$
				\item \circled{17} $a = b \supset a^{\backprime} = b^{\backprime}$
				\item \circled{18} $a + 0 = a$
				\item \circled{19} $a + b^{\backprime} = (a + b)^{\backprime}$
				\item \circled{20} $a \cdot 0 = 0$
				\item \circled{21} $a \cdot b^{\backprime} = a \cdot b + a$
				\item[\vspace{\fill}]				
				\end{itemize}
			\end{multicols}
		
		\begin{definition}[\textsection 19]
			Формула является \textit{аксиомой}, если она имеет форму одну из \circled{1a}, 
			\circled{1b}, \circled{3}--\circled{8}, \circled{10}, \circled{11}, \circled{13} или она
			есть одна из \circled{14}--\circled{21}.  		
		\end{definition}
	
		\begin{definition}[\textsection 19]
			Формула является \textit{непосредственным следствием} (из) одной или двух других формул,
			если она имеет форму, указанную под чертой, тогда как другая (не) имеет(ют) форму(ы),
			указанную (не) над чертой в \circled{2}, \circled{9} или \circled{12}.    		
		\end{definition}
	
		\begin{definition}[\textsection 19]
			Постулаты \circled{2}, \circled{9} и \circled{12} мы называем 
			\textit{правилами вывода}. Для любого (фиксированного) выбора $A$ и $B$ или $x$, $A(x)$
			и $C$, подчинённого отмеченным выше условиям, формулы указанные над чертой, являются
			\textit{посылкой} (\textit{первой} и \textit{второй посылкой} соответственно), а 
			формула, указанная под чертой, является \textit{заключением} применения правила вывода.    		
		\end{definition}
		
	\subsection*{Формальный вывод}
	
	\begin{definition}[\textsection 20]
		Если дан перечень $D_1, \dots, D_l(l\ge0)$ формул, то непустая конечная последовательность формул
		называется \textit{формальным выводом из исходных формул} $D_1, \dots, D_l$, если каждая формула
		этой последовательности является или одной из формул $D_1, \dots, D_l$, или аксиомой, или
		непосредственным следствием из предыдущих формул последовательности. Вывод называется выводом
		\textit{своей последней} формулы $E$, и эта формула называется \textit{выводимой из} исходных
		формул (обозначается $D_1, \dots, D_l \vdash E$), а также \textit{заключением} (или 
		\textit{конечной формулой}) вывода.
	\end{definition}
	
	\begin{definition}[\textsection 20, Общие свойства $\vdash$]
		\leavevmode
		\begin{itemize}
			\setlength\itemsep{-3pt}
			\item $\Gamma \vdash E$, если $E$ входит в список $\Gamma$
			\item Если $\Gamma \vdash E$, то $\Delta, \Gamma \vdash E$ для любого перечня $\Delta$
				(Любая доказуемая выводима из любых исходных)
			\item Если $\Gamma \vdash E$, то $\Delta \vdash E$, где $\Delta$ получается из $\Gamma$
				путём перестановки формул $\Gamma$ или опускания любых таких формул, которые 
				тождественны с другими остающимися
			\item Если $\Gamma \vdash E$, то $\Delta \vdash E$, где $\Delta$ получается из $\Gamma$
				опусканием любых формул $\Gamma$, которые являются доказуемыми или выводимыми из
				остающихся формул $\Gamma$.
		\end{itemize}
	\end{definition}
	
	\begin{theorem}[\textsection 21, О дедукции]
		Для исчисления высказываний, если $\Gamma, A \vdash B$, то 
		$\Gamma \vdash A \supset B$.
	\end{theorem}

	\setcounter{theorem}{0}
	
	\begin{theorem}[\textsection 22, полная]
		 Для исчисления предикатов (или полной арифметической формальной системы), если 
		$\Gamma, A \vdash B$, причём все свободные переменные остаются фиксированными 
		для последней исходной формулы, то $\Gamma \vdash A \supset B$.
	\end{theorem}

	\begin{definition}[\textsection 23]
		Переменная ``$x$'' приписанная к символу ``$\vdash$'' в качестве верхнего индекса
		отличает применение правила \circled{9} или \circled{12} по отношению к $x$ 
		при построении результирующего вывода.		
	\end{definition}

	\begin{theorem}[\textsection 23]
		В следующих правилах $A$, $B$ и $C$ или $x$, $A(x)$, $C$ и $t$ подчинены тем же 
		условиям, что и в соответствующих постулатах, а $\Gamma$ или $\Gamma(x)$ есть 
		любой список формул. 
		
		Для исчисления высказываний справедливы
		правила от ``импликации'' до ``отрицания'' включительно.
		
		Для исчисления предикатов (или полной арифметической системы) справедливы все 
		правила, при условии, что в каждом вспомогательном выводе связанные переменные
		остаются фиксированными для устраняемой формулы.
		
		\begin{center}
			\begin{tabular}{ | m{3cm} | m{5.2cm}| m{5cm} | }
				\hline
							& Введение & Удаление \\ 
				\hline
				Импликация 	& Если $\Gamma, A \vdash B$, \newline то $\Gamma \vdash A 
							\supset B$    
							& $A, A \supset B \vdash B$ \newline (modus ponens) \\
				\hline
				Конъюнкция 	& $A, B \vdash A \conj B$ 
							& $A \conj B \vdash A$ \newline $A \conj B \vdash B$ \\
				\hline
				Дизъюнкция 	& $A \vdash A \vee B$ \newline $B \vdash A \vee B$ 
							& Если $\Gamma, A \vdash C$ и $\Gamma, B \vdash C$, \newline то
							$\Gamma, A \vee B \vdash C$ \\
				\hline
				Отрицание  	& Если $\Gamma, A \vdash B$ и $\Gamma, A \vdash \neg B$, 
							\newline то $\Gamma \vdash \neg A$ 
							& $\neg \neg A \vdash A$ \\
				\hline
				Общность   	& $A(x) \vdash^x \forall x A(x)$ 
							& $\forall x A(x) \vdash A(t)$ \\
				\hline
				Существование & $A(t) \vdash \exists x A(x)$ 
							& Если $\Gamma(x), A(x) \vdash C$, \newline то $\Gamma(x), \exists x A(x) \vdash^x C$ \\
				\hline
			\end{tabular}
		\end{center}  
	\end{theorem}

	\newpage
	
	\subsection*{Формулы исчисления высказываний}
	
	\begin{definition}[\textsection 25]
		Формальные символы нового рода: $\mathscr{A}, \mathscr{B}, \mathscr{C}, \dots$ называемые
		пропозициональными буквами, (потенциально) бесконечный перечень которых мы считаем
		имеющимся в нашем распоряжении. Новое определение ``формулы'': 
		\begin{enumerate}
			\setlength\itemsep{-3pt}	
			\item Пропозициональная буква есть \textit{формула}
			\item Если $A$ и $B$ --- \textit{формулы}, то
			\begin{multicols}{3}
				\begin{enumerate}
					\item $(A) \supset (B)$ --- \textit{формула}
					\item $(A) \conj\: (B)$ --- \textit{формула}
					\item $(A) \vee (B)$    --- \textit{формула}
					\item $\neg (A)$        --- \textit{формула}
				\end{enumerate}
			\end{multicols}
			\item Никаких других \textit{формул}, кроме определённых согласно 1 и 2, нет.
		\end{enumerate}
	\end{definition}

	\begin{definition}[\textsection 25]
		Пусть $P_1, \dots, P_m$ --- перечень различных пропозициональных букв. (Здесь 
		$``P_1"$$ , \dots, ``P_m"$ --- метаматематические буквы, которыми мы пользуемся как
		названиями для пропозициональных букв, когда не хотим ограничивать наше рассуждение
		употреблением конкретных пропозициональных букв.)
		
		Пропозициональная формула $A$ называется \textit{формулой составленной} из 
		$P_1, \dots, P_m$, если никакая пропозициональная буква, отличная от $P_1, \dots, P_m$ 
		не входит в $A$.
	\end{definition}

	\begin{definition}[\textsection 25]
		\textit{Подстановка} вместо пропозициональной буквы (или одновременно вместо
		нескольких различных пропозициональных пропозициональных букв) определяется как для
		переменной в \textsection 18, за исключением того, что подстановка применяется теперь
		ко всем вхождениям без исключений (так как нет связанных вхождений).
	\end{definition}

	\begin{theorem}[\textsection 25, Подстановка вместо пропозициональных букв]
		Пусть $\Gamma$ --- перечень пропозициональных формул, а $E$ --- пропозициональная формула, 
		составленная из различных пропозициональных букв $P_1, \dots, P_m$. Пусть 
		$A_1, \dots, A_m$ --- формулы. Пусть $\Gamma^*$ и $E^*$ получаются из $\Gamma$ и $E$ 
		соответственно путём одновременной подстановки $A_1, \dots, A_m$ вместо $P_1, \dots, P_m$ 
		соответственно. 
		
		Если $\Gamma \vdash E$, то $\Gamma^* \vdash E^*$ (Для случая пустой $\Gamma$: если 
		$\vdash E$, то $\vdash E^*$)
	\end{theorem}

	\begin{definition}[\textsection 25]
		Формула называется \textit{элементарной (для исчисления высказываний)}, если она не 
		имеет ни одного из видов $A \supset B$, $A \conj B$, $A \vee B$, $\neg A$, где $A$
		и  $B$ --- формулы.		
	\end{definition}

	\begin{theorem}[\textsection 25, Обращение правила подстановки для проп. переменных]
		При тех же условиях, что и в теореме 3. Если $A_1, \dots, A_m$ --- элементарные формулы, то
		из $\Gamma^* \vdash E^*$ следует $\Gamma \vdash E$.
	\end{theorem}

	\setcounter{theorem}{3}
	
	\begin{theorem}[\textsection 25, вторая форма]
		Пусть $\Gamma^*$ --- формулы, а $E^*$ --- формула, имеющие различные элементарные 
		компоненты $A_1, \dots, A_m$. Пусть $P_1, \dots, P_m$ --- пропозициональные буквы, не 
		обязательно различные. Пусть $\Gamma$, $E$ получаются из $\Gamma^*$, $E^*$ соответственно 
		заменой одновременно во всех вхождениях $A_1, \dots, A_m$ на $P_1, \dots, P_m$ 
		соответственно. Тогда $\Gamma^* \vdash E^*$ влечёт $\Gamma \vdash E$.
		
		За исключением того, что ``формула'' здесь понимается не в смысле пропозициональной 
		формулы, теорема 4 содержится в теореме 3.
	\end{theorem}

	\begin{definition}[\textsection 26]
		Пусть $A$ и $B$ --- формулы. Введём запись ``$A \sim B$'' в качестве сокращения для записи
		$(A \supset B) \conj\ (B \supset A)$. Символ ``$\sim$'' можно читать ``эквивалентна''. Он
		употребляется в качестве формального оператора, который, будучи помещён между двумя
		формулами системы, даёт другую формулу этой системы. При опускании скобок ему
		приписывается ранг более высокий, чем другим формальным операторам (\textsection 17)
		
		$A$ \textit{эквивалентна} $B$ в исчислении высказываний или в другой формальной система,
		если в этой формальной системе $\vdash A \sim B$. Здесь слово ``эквивалентна''
		употребляется в качестве метаматематического глагола, который, будучи помещён между 2
		формулами системы, даёт высказывание об этих формулах 
	\end{definition}

	\newpage
	\begin{theorem}[\textsection 26]
		Если $A$, $B$ и $C$ --- формулы, то:
		\begin{itemize}[label={}]
			\setlength\itemsep{0pt}	
			\item 1. $\vdash A \supset A$ --- принцип тождества
			\item 2. $A \supset B, B \supset C \vdash A \supset C$ --- 
				цепное заключение
			\item 3. $A \supset (B \supset C) \vdash B \supset (A \supset C)$ ---
				перестановка посылок
			\item 4. $A \supset (B \supset C) \vdash A \conj B \supset C$ ---
				импортация
			\item 5. $A \conj B \supset C \vdash A \supset (B \supset C)$ ---
				экспортация
		\end{itemize}
		
		Введение в импликацию
		\begin{itemize}[label={}]
			\setlength\itemsep{0pt}	
			\item 6. $A \supset B \vdash (B \supset C) \supset (A \supset C)$ --- заключения
			\item 7. $A \supset B \vdash (C \supset A) \supset (C \supset B)$ --- посылки
			\item 8a. $A \supset B \vdash A \conj C \supset B \conj C$ ---
				конъюнктивного члена
			\item 8b. $A \supset B \vdash C \conj A \supset C \conj B$
			\item 9a. $A \supset B \vdash A \vee C \supset B \vee C$ ---
				дизъюнктивного члена
			\item 9b. $A \supset B \vdash C \vee A \supset C \vee A$ 
		\end{itemize}
	
		Доказательство импликации путём
		\begin{itemize}[label={}]
			\setlength\itemsep{0pt}	
			\item 10a. $\neg A \vdash A \supset B$ --- опровержения посылки
			\item 10b. $A \vdash \neg A \supset B$
			\item 11. $B \vdash A \supset B$ --- доказательства заключения
		\end{itemize}
	
		Контрапозиция
		\begin{itemize}[label={}]
			\setlength\itemsep{0pt}	
			\item 12. $A \supset B \vdash \neg B \supset \neg A$
			\item 13. $A \supset \neg B \vdash B \supset \neg A$
			
			со снятием двойного отрицания
			\item 14$^{\circ}$. $\neg A \supset B \vdash \neg B \supset A$
			\item 15$^{\circ}$. $\neg A \supset \neg B \vdash B \supset A$
		\end{itemize}
	
		По определению $\sim$ в терминах $\supset$ и $\conj$
		\begin{itemize}[label={}]
			\setlength\itemsep{0pt}	
			\item 16. $A \supset B, B \supset A \vdash A \sim B$
			\item 17a. $A \sim B \vdash A \supset B$
			\item 17b. $A \sim B \vdash B \supset A$
			\item 18a. $A \sim B, A \vdash B$
			\item 18b. $A \sim B, B \vdash A$
			\item 19. $\vdash A \sim A$ --- рефлексивность
			\item 20. $A \sim B \vdash B \sim A$ --- симметричность
			\item 21. $A \sim B, B \sim C \vdash A \sim C$ --- транзитивность
		\end{itemize}
	
	
		Дополнительные результаты, представляющие интерес в связи с интуиционистской системой
		\begin{itemize}[label={}]
			\setlength\itemsep{0pt}	
			\item 22. $A \supset (B \supset C), \neg \neg A, \neg \neg B \vdash \neg \neg C$
			\item 23. $\neg \neg (A \supset B) \vdash \neg \neg A \supset \neg \neg B$
			\item 24. $\neg \neg (A \supset B), \neg \neg (B \supset C) \vdash \neg \neg (A \supset C)$
			\item 25. $\vdash \neg \neg (A \conj B) \sim \neg \neg A \conj \neg \neg B$; в частности
				$\vdash \neg \neg (A \sim B) \sim \neg \neg (A \supset B) \conj \neg \neg (B \supset A)$
		\end{itemize}
	\end{theorem}
	
	\begin{definition}[\textsection 26]
		Пусть $A$ --- формальное выражение. Рассмотрим другое формальное выражение $C$. Может
		случиться, что $A$ входит в $C$ как (связная) часть, причём это возможно более чем одним
		способом. Допустим, что это имеет место и что, если это осуществляется более чем одним
		способом, то выделено некоторое конкретное вхождение $A$ в $C$. Обозначим теперь $C$ вместе
		с выделенным конкретным вхождением $A$ в $C$ через ``$C_A$''. В обозначении сочленения
		$C_A$ есть $EAF$, где $E$ и $F$ --- части (возможно, пустые), предшествующая и следующая за
		этой выделенной частью $A$. Пусть теперь $B$ --- какое-то формальное выражение. Результатом
		\textit{замены} этой выделенной части $A$ выражения $C$ на $B$ есть выражение $EBF$.
		Обозначим через $C_B$. 
	\end{definition}

	\begin{theorem}[\textsection 26, Теорема о замене]
		Если $A$, $B$, $C_A$ и $C_B$ --- пропозициональные формулы, связанные друг с другом, как в
		предыдущем определении замены, то		
		\begin{center}
			$A \sim B \vdash C_A \sim C_B$
		\end{center}
	\end{theorem}
	
	\setcounter{theorem}{5}
	
	\begin{theorem}[\textsection 26, Леммы для замены]
		Если $A$, $B$, $C$ ---  формулы, то
		\begin{multicols}{2}		
			\begin{itemize}[label={}]
				\setlength\itemsep{0pt}	
				\item 26. $A \sim B \vdash A \supset C \sim B \supset C$
				\item 27. $A \sim B \vdash C \supset A \sim C \supset B$
				\item 28a. $A \sim B \vdash A \conj C \sim B \conj C$
				\item 28b. $A \sim B \vdash C \conj A \sim C \conj B$
				\item 29a. $A \sim B \vdash A \vee C \sim B \vee C$
				\item 29b. $A \sim B \vdash C \vee A \sim C \vee B$
				\item 30. $A \sim B \vdash \neg A \sim \neg B$
				\item[\vspace{\fill}]
			\end{itemize}
		\end{multicols}
	\end{theorem}

	\setcounter{theorem}{5}
	
	\begin{theorem}[\textsection 26, вторая форма]
		Если $A$ и $B$ --- формулы, $C_A$ --- формула, построенная из некоторого конкретного
		вхождения $A$ с помощью одних только операторов $\supset$, $\conj$, $\vee$, $\neg$, а
		 $C_B$ получается из $C_A$ заменой этого вхождения $A$ на $B$, то		
		\begin{center}
			$A \sim B \vdash C_A \sim C_B$
		\end{center}
	\end{theorem}

	\setcounter{theorem}{5}
	
	\begin{theorem}[\textsection 26, Следствие: свойство замены для эквивалентности]
		В условиях теоремы (в любой форме)		
		\begin{center}
			$A \sim B, C_A \vdash C_B$
		\end{center}
	\end{theorem}

	\begin{theorem}[\textsection 27]
		%%
		%% TODO: аналогично теореме 5
		%%
	\end{theorem}

	\begin{theorem}[\textsection 27, используется постулат \circled{$8^{\circ}$}]
		Пусть $D$ --- пропозициональная формула, построенная из различных пропозициональных букв
		$P_1,\dots, P_m$ и их отрицаний $\neg P_1, \dots, \neg P_m$ с помощью одних только
		операторов $\conj$, $\vee$. Тогда формула $D^{\dagger}$, эквивалентная $\neg D$,
		получается в результате замены друг на друга в $D$ символов $\conj$ и $\vee$ и ещё каждой
		буквы и её отрицания.
		
		Другими словами, если $D$ --- пропозициональная формула описанного рода, а $D^{\dagger}$
		--- результат описанной замены друг на друга в $D$, то 
		\begin{center}
			$\vdash \neg D \sim D^{\dagger}$
		\end{center}
	\end{theorem}

	\setcounter{theorem}{7}
	
	\begin{theorem}[\textsection 27, Следствие: принцип двойственности]
		Эквивалентность между двумя \textit{формулами $E$ и $F$ описанного в теореме 8
		типа} сохраняется при замене друг на друга в $E$ и $F$ символов $\conj$ и $\vee$.
		
		Другими словами, если $E$ и $F$ --- две такие пропозициональные формулы, а $E'$ и $F'$
		получаются в результате указанной замены друг на друга в $E$ и $F$ соответственно, то
		\begin{center}
			из $\vdash E \sim F$ следует $\vdash E' \sim F'$
		\end{center}
	\end{theorem}

	\setcounter{theorem}{7}
	
	\begin{theorem}[\textsection 27, Следствие: вторая часть, соотношение обратной двойственности]
		При тех же условиях из $\vdash E \supset F$ следует $\vdash F' \supset E'$
	\end{theorem}

	\subsection*{Оценка, непротиворечивость}
	
	\begin{definition}[\textsection 28]
		Исчисление высказываний (и вообще любая формальная система, имеющая символ $\neg$ для
		отрицания) называется \textit{(просто) непротиворечивой} системой, если ни для какой
		формулы $A$ и $\neg A$ не являются обе доказуемыми в этой системе, и \textit{(просто)
		противоречивой} в противном случае, если для некоторой формулы $A$ одновременно
		$\vdash A$ и $\vdash \neg A$.
	\end{definition}

	Это строго метаматематическое определение. Оно опирается только на формальный символ $\neg$ и
	определения формулы и доказуемой формулы. Таким образом, доказательство непротиворечивости
	данной формальной системы становится точной математической проблемой, которую можно
	рассматривать в метаматематике.
	
	Метаматематическое доказательство непротиворечивости формальной системы даёт гарантию против
	возникновения противоречия в соответствующей содержательной теории.
	
	Для исчисления высказываний (и вообще любой формальной системы, которая содержит
	$\conj$-удаление и слабое $\neg$-удаление в качестве постулируемых или выводимых правил),
	предыдущее определение эквивалентно следующему
	
	\begin{definition}[\textsection 28]
		Система \textit{(просто) непротиворечива}, если в ней имеется некоторая недоказуемая
		формула; \textit{(просто) противоречива}, если любая формула доказуема.
	\end{definition}

	Предположим, что мы нашли метаматематическое свойство формул, такое, что
	\begin{itemize}[label={}]
		\setlength\itemsep{0pt}	
		\item \circled{a} аксиомы обладают этим свойством
		\item \circled{b} при каждом применении правила вывода, если посылки обладают этим
						 свойством, то и заключение тоже
		\item \circled{c} две формулы вида $A$ и $\neg A$ не могут обе обладать этим свойством.
	\end{itemize}

	Тогда в силу \circled{a} и \circled{b} каждая доказуемая формула будет обладать этим свойством и в силу \circled{c} система оказывается непротиворечивой.
	
	Теперь построим некоторую арифметику для области только с двумя предметами и четырьмя
	функциями $\supset$, $\conj$, $\vee$, $\neg$. Поскольку для метаматематики $\supset$, $\conj$,
	$\vee$, $\neg$ являются объектами, не имеющими смысла, более точное описание состоит в
	следующем. 
	
	\begin{definition}[\textsection 28]
		Введём некоторый метаматематический вычислительный процесс (называемый \textit{проведением
		оценки}), согласно которому с каждым из символов $\supset$, $\conj$, $\vee$, $\neg$ будет
		связана некоторая функция из этой арифметики (или таблица для такой функции, называемая
		\textit{таблицей истинности}).
	\end{definition}

	Тем самым с каждой пропозициональной формулой будет связана некоторая такая функция. Затем мы
	изучим метаматематические свойства пропозициональных формул, определённые в терминах
	соответствующих функций (или таблиц).
	
	\begin{multicols}{4}
		\begin{tabular}{ | c | c | c | }
			\hline
			$\mathscr{A}$ & $\mathscr{B}$ & $\mathscr{A} \supset \mathscr{B}$ \\ 
			\hline
			$f$ & $f$ & $t$ \\
			\hline
			$f$	& $t$ & $t$ \\
			\hline
			$t$	& $f$ & $f$\\
			\hline
			$t$ & $t$ & $t$ \\
			\hline
		\end{tabular}

		\begin{tabular}{ | c | c | c | }
			\hline
			$\mathscr{A}$ & $\mathscr{B}$ & $\mathscr{A} \conj \mathscr{B}$ \\ 
			\hline
			$f$ & $f$ & $f$ \\
			\hline
			$f$	& $t$ & $f$ \\
			\hline
			$t$	& $f$ & $f$\\
			\hline
			$t$ & $t$ & $t$ \\
			\hline
		\end{tabular}
	
		\begin{tabular}{ | c | c | c | }
			\hline
			$\mathscr{A}$ & $\mathscr{B}$ & $\mathscr{A} \vee \mathscr{B}$ \\ 
			\hline
			$f$ & $f$ & $f$ \\
			\hline
			$f$	& $t$ & $t$ \\
			\hline
			$t$	& $f$ & $t$\\
			\hline
			$t$ & $t$ & $t$ \\
			\hline
		\end{tabular}
	
		\begin{tabular}{ | c | c | }
			\hline
			$\mathscr{A}$ & $\neg \mathscr{A}$ \\ 
			\hline
			$f$ & $t$  \\
			\hline
			$t$	& $f$  \\
			\hline
		\end{tabular}
	\end{multicols}
	
	Тогда каждая пропозициональная формула $A$, составленная из данного перечня различных
	пропозициональных букв $P_1, \dots, P_m$, представляется как функция от этих букв,
	рассмотренные как независимые переменные над областью $\{t, f\}$. Для каждой $m$-ки значений
	этих букв соответствующее значение функции может быть вычислено путём последовательного
	применения этих основных таблиц.
	
	\begin{definition}[\textsection 28]
		Пропозициональная формула $E$, составленная из различных пропозициональных букв $P_1,
		\dots, P_m$, называется \textit{тождественно истинной}, если столбец значений её таблицы
		содержит одни только $t$, и \textit{тождественно ложной}, если он содержит одни только $f$.
		
		Две пропозициональные формулы $E$ и $F$ составленные из $P_1, \dots, P_m$, называются
		\textit{тождественно равными}, если их таблицы имеют один и тот же столбец значений. 
	\end{definition}

	\begin{theorem}[\textsection 28]
		Для того, чтобы пропозициональная формула $E$ была доказуемой (или выводимой из
		тождественно истинных формул $\Gamma$) в исчислении высказываний, необходимо, чтобы она
		была тождественно истиной, т.е.
		\begin{center}
			если $\vdash E$, то $E$ тождественно истинна.
		\end{center}
	\end{theorem}

	\setcounter{theorem}{8}
	
	\begin{theorem}[\textsection 28, Следствие 1]
		Для того, чтобы две пропозициональные формулы $E$ и $F$ были эквивалентны, необходимо,
		чтобы они были тождественно равны:
		\begin{center}
			если $\vdash E \sim F$, то $E$ и $F$ тождественно равны.
		\end{center}
	\end{theorem}

	\setcounter{theorem}{8}
	
	\begin{theorem}[\textsection 28, Следствие 2]
		Исчисление высказываний (просто) непротиворечиво; другими словами, ни для какой формулы
		$A$ не имеют место одновременно $\vdash A$ и $\vdash \neg A$.
	\end{theorem}
\end{document}