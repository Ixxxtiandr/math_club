\documentclass[a4paper, 12pt]{article}  % добавить leqno в [] для нумерации слева

%%% Работа с русским языком
\usepackage{cmap}
\usepackage[utf8]{inputenc}
\usepackage[english, russian]{babel}
\usepackage{pscyr}

%%% Дополнительная работа с математикой
\usepackage{amsmath,amsfonts,amssymb,amsthm,mathtools} % AMS
\usepackage{icomma} % "Умная" запятая

%% Номера формул
\mathtoolsset{showonlyrefs=true} % Показывать номера только у тех формул, на которые есть \eqref{} в тексте.

%% Шрифты
\usepackage{euscript} % Шрифт Евклид
\usepackage{mathrsfs} % Красивый матшрифт

%% Первый абзац после секшиона
\usepackage{indentfirst}

%% Свои команды
\DeclareMathOperator{\pphi}{\mathop{\varphi}}
\DeclareMathOperator{\eps}{\mathop{\varepsilon}}
\DeclareMathOperator{\conj}{\mathbb{\&}}

%% Перенос знаков в формулах (по Львовскому)
\newcommand*{\hm}[1]{#1\nobreak\discretionary{}
	{\hbox{$\mathsurround=0pt #1$}}{}}

%% Поля
\usepackage[left=1cm, right=1.4cm, top=1cm, bottom=1.3cm]{geometry}

%% Графика
\usepackage{tikz}

%% Переименовать список литературы
\addto\captionsrussian{\def\refname{Литература}}

%%% Заголовок
\author{Мат.клуб ``Тифаретник'' по С. Клини}
\title{Введение в метаматематику на троечку}
\date{\today}

%%% Эпиграф
\usepackage{epigraph}
\setlength\epigraphwidth{9cm}

%%% Работа с картинками
\usepackage{graphicx}  % Для вставки рисунков
\graphicspath{{images/}{images2/}}  % папки с картинками
\setlength\fboxsep{3pt} % Отступ рамки \fbox{} от рисунка
\setlength\fboxrule{1pt} % Толщина линий рамки \fbox{}
\usepackage{wrapfig} % Обтекание рисунков и таблиц текстом

%%% Добавляем код. Окружение lstlisting
\usepackage{listings}
\usepackage{color}
\definecolor{mygreen}{rgb}{0,0.6,0}
\definecolor{bggray}{RGB}{244,244,244}
\definecolor{mygray}{rgb}{0.5,0.5,0.5}
\definecolor{mymauve}{rgb}{0.58,0,0.82}
\lstset
{
	backgroundcolor=\color{bggray},   % choose the background color; you must add \usepackage{color} or \usepackage{xcolor}; should come as last argument
	basicstyle=\footnotesize,        % the size of the fonts that are used for the code
	breakatwhitespace=false,         % sets if automatic breaks should only happen at whitespace
	breaklines=true,                 % sets automatic line breaking
	captionpos=b,                    % sets the caption-position to bottom
	commentstyle=\color{mygreen},    % comment style
	deletekeywords={...},            % if you want to delete keywords from the given language
	escapeinside={\%*}{*)},          % if you want to add LaTeX within your code
	extendedchars=true,              % lets you use non-ASCII characters; for 8-bits encodings only, does not work with UTF-8
	%	firstnumber=1000,                % start line enumeration with line 1000
	%	frame=single,	                   % adds a frame around the code
	keepspaces=true,                 % keeps spaces in text, useful for keeping indentation of code (possibly needs columns=flexible)
	keywordstyle=\color{blue},       % keyword style
	language=c,                 	 % the language of the code
	morekeywords={*,...},            % if you want to add more keywords to the set
	numbers=none,                    % where to put the line-numbers; possible values are (none, left, right)
	numbersep=5pt,                   % how far the line-numbers are from the code
	numberstyle=\tiny\color{mygray}, % the style that is used for the line-numbers
	rulecolor=\color{black},         % if not set, the frame-color may be changed on line-breaks within not-black text (e.g. comments (green here))
	showspaces=false,                % show spaces everywhere adding particular underscores; it overrides 'showstringspaces'
	showstringspaces=false,          % underline spaces within strings only
	showtabs=false,                  % show tabs within strings adding particular underscores
	stepnumber=2,                    % the step between two line-numbers. If it's 1, each line will be numbered
	stringstyle=\color{mymauve},     % string literal style
	tabsize=3,	                     % sets default tabsize to 2 spaces
	title=\lstname                   % show the filename of files included with \lstinputlisting; also try caption instead of title	
}

%%% Ссылки
\usepackage{hyperref}

%%% Окружения
\theoremstyle{definition}
\newtheorem{theorem}{Теорема}
\newtheorem*{definition}{Определение}
\newtheorem*{pr}{Доказательство}

%%% Цифры в кружке
\newcommand*\circled[1]{\tikz[baseline=(char.base)]{
		\node[shape=circle,draw,inner sep=2pt] (char) {#1};}}

%%% Перечисление кастомное
\usepackage{enumitem}

%%% Естественный вывод
\usepackage{bussproofs}

%%% Несколько колонок для перечислений
\usepackage{multicol}

\usepackage{array}

%%% Без числа страниц
\pagenumbering{gobble}

\begin{document}
	{
		\centering
		\section*{Постулаты формальной системы}
	}
	\subsubsection*{Dramatis personae}
	В постулатах 1--8 $A, B$ и $C$ --- формулы. В 9--13 $x$ --- переменная, $A(x)$ --- формула, 
	$C$ --- формула, не содержащая свободно $x$, а $t$ --- терм, свободный для $x$ в $A(x)$.
	
	\subsection*{Группа А. Постулаты исчисления предикатов}
	
	\subsubsection*{Группа А1. Постулаты исчисления высказываний}
	\begin{multicols}{2}
		\begin{itemize}[label={}]
			\item \circled{1a} $A \supset (B \supset A)$
			\item \circled{1b} $(A \supset B) \supset ((A \supset (B \supset C)) \supset (A \supset C))$
			\item \circled{4a} $A \conj B \supset A$
			\item \circled{4b} $A \conj B \supset B$
			\item \circled{5a} $A \supset A \vee B$
			\item \circled{5b} $B \supset A \vee B$
			\item \circled{2} 
			\AxiomC{$A, A \supset B $}
			\UnaryInfC{$B$}
			\DisplayProof
			\item \circled{3} $A \supset (B \supset A \conj B)$
			\item \circled{6} $(A \supset C) \supset ((B \supset C) \supset (A \vee B \supset C))$
			\item \circled{7} $(A \supset B) \supset ((A \supset \neg B) \supset \neg A)$
			\item \circled{$8^{\circ}$} $\neg \neg A \supset A$
			\item[\vspace{\fill}]
		\end{itemize}
	\end{multicols}
	
	\subsubsection*{Группа А2. (Дополнительные) Постулаты исчисления предикатов}
	\begin{multicols}{2}
		\begin{itemize}[label={}]
			\item \circled{9}
			\AxiomC{$C \supset A(x)$}
			\UnaryInfC{$C \supset \forall x A(x)$}
			\DisplayProof
			\item \circled{10} $\forall x A(x) \supset A(t)$
			\item \circled{11} $A(t) \supset \exists x A(x)$
			\item \circled{12} 
			\AxiomC{$A(x) \supset C$}
			\UnaryInfC{$\exists x A(x) \supset C$}
			\DisplayProof
		\end{itemize}
	\end{multicols}
	
	\subsection*{Группа B. (Дополнительные) Постулаты арифметики}
	\begin{multicols}{2}
		\begin{itemize}[label={}]
			\setlength\itemsep{0pt}	
			\item \circled{13} $A(0) \conj \forall x (A(x) \supset A(x^{\backprime})) \supset A(x)$
			\item \circled{14} $a^{\backprime} = b^{\backprime} \supset a = b$
			\item \circled{15} $\neg a^{\backprime} = 0$
			\item \circled{16} $a = b \supset (a = c \supset b = c)$
			\item \circled{17} $a = b \supset a^{\backprime} = b^{\backprime}$
			\item \circled{18} $a + 0 = a$
			\item \circled{19} $a + b^{\backprime} = (a + b)^{\backprime}$
			\item \circled{20} $a \cdot 0 = 0$
			\item \circled{21} $a \cdot b^{\backprime} = a \cdot b + a$
			\item[\vspace{\fill}]				
		\end{itemize}
	\end{multicols}

	\begin{definition}[\textsection 19]
		Формула является \textit{аксиомой}, если она имеет форму одну из \circled{1a}, 
		\circled{1b}, \circled{3}--\circled{8}, \circled{10}, \circled{11}, \circled{13} или она
		есть одна из \circled{14}--\circled{21}.  		
	\end{definition}
	
	\begin{definition}[\textsection 19]
		Формула является \textit{непосредственным следствием} (из) одной или двух других формул,
		если она имеет форму, указанную под чертой, тогда как другая (не) имеет(ют) форму(ы),
		указанную (не) над чертой в \circled{2}, \circled{9} или \circled{12}.    		
	\end{definition}
	
	\begin{definition}[\textsection 19]
		Постулаты \circled{2}, \circled{9} и \circled{12} мы называем 
		\textit{правилами вывода}. Для любого (фиксированного) выбора $A$ и $B$ или $x$, $A(x)$
		и $C$, подчинённого отмеченным выше условиям, формулы указанные над чертой, являются
		\textit{посылкой} (\textit{первой} и \textit{второй посылкой} соответственно), а 
		формула, указанная под чертой, является \textit{заключением} применения правила вывода.    		
	\end{definition}
\end{document}