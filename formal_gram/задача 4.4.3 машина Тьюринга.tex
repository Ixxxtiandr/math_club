\documentclass[a4paper, 12pt]{article}  % добавить leqno в [] для нумерации слева

%%% Работа с русским языком
\usepackage{cmap}
\usepackage[utf8]{inputenc}
\usepackage[english, russian]{babel}
\usepackage{pscyr}

%%% Дополнительная работа с математикой
\usepackage{amsmath,amsfonts,amssymb,amsthm,mathtools} % AMS
\usepackage{icomma} % "Умная" запятая

%% Номера формул
\mathtoolsset{showonlyrefs=true} % Показывать номера только у тех формул, на которые есть \eqref{} в тексте.

%% Шрифты
\usepackage{euscript} % Шрифт Евклид
\usepackage{mathrsfs} % Красивый матшрифт

%% Первый абзац после секшиона
\usepackage{indentfirst}

%% Свои команды
\DeclareMathOperator{\pphi}{\mathop{\varphi}}
\DeclareMathOperator{\eps}{\mathop{\varepsilon}}
\DeclareMathOperator{\conj}{\mathbb{\&}}

%% Перенос знаков в формулах (по Львовскому)
\newcommand*{\hm}[1]{#1\nobreak\discretionary{}
	{\hbox{$\mathsurround=0pt #1$}}{}}

%% Поля
\usepackage[left=1cm, right=1.4cm, top=1cm, bottom=1.3cm]{geometry}

%% Графика
\usepackage{tikz}
\usetikzlibrary{automata,positioning}

%% Переименовать список литературы
\addto\captionsrussian{\def\refname{Литература}}

%%% Заголовок
\author{Мат.клуб ``Тифаретник'' по С. Клини}
\title{Введение в метаматематику на троечку}
\date{\today}

%%% Эпиграф
\usepackage{epigraph}
\setlength\epigraphwidth{9cm}

%%% Работа с картинками
\usepackage{graphicx}  % Для вставки рисунков
\graphicspath{{images/}{images2/}}  % папки с картинками
\setlength\fboxsep{3pt} % Отступ рамки \fbox{} от рисунка
\setlength\fboxrule{1pt} % Толщина линий рамки \fbox{}
\usepackage{wrapfig} % Обтекание рисунков и таблиц текстом

%%% Добавляем код. Окружение lstlisting
\usepackage{listings}
\usepackage{color}
\definecolor{mygreen}{rgb}{0,0.6,0}
\definecolor{bggray}{RGB}{244,244,244}
\definecolor{mygray}{rgb}{0.5,0.5,0.5}
\definecolor{mymauve}{rgb}{0.58,0,0.82}
\lstset
{
	backgroundcolor=\color{bggray},   % choose the background color; you must add \usepackage{color} or \usepackage{xcolor}; should come as last argument
	basicstyle=\footnotesize,        % the size of the fonts that are used for the code
	breakatwhitespace=false,         % sets if automatic breaks should only happen at whitespace
	breaklines=true,                 % sets automatic line breaking
	captionpos=b,                    % sets the caption-position to bottom
	commentstyle=\color{mygreen},    % comment style
	deletekeywords={...},            % if you want to delete keywords from the given language
	escapeinside={\%*}{*)},          % if you want to add LaTeX within your code
	extendedchars=true,              % lets you use non-ASCII characters; for 8-bits encodings only, does not work with UTF-8
	%	firstnumber=1000,                % start line enumeration with line 1000
	%	frame=single,	                   % adds a frame around the code
	keepspaces=true,                 % keeps spaces in text, useful for keeping indentation of code (possibly needs columns=flexible)
	keywordstyle=\color{blue},       % keyword style
	language=c,                 	 % the language of the code
	morekeywords={*,...},            % if you want to add more keywords to the set
	numbers=none,                    % where to put the line-numbers; possible values are (none, left, right)
	numbersep=5pt,                   % how far the line-numbers are from the code
	numberstyle=\tiny\color{mygray}, % the style that is used for the line-numbers
	rulecolor=\color{black},         % if not set, the frame-color may be changed on line-breaks within not-black text (e.g. comments (green here))
	showspaces=false,                % show spaces everywhere adding particular underscores; it overrides 'showstringspaces'
	showstringspaces=false,          % underline spaces within strings only
	showtabs=false,                  % show tabs within strings adding particular underscores
	stepnumber=2,                    % the step between two line-numbers. If it's 1, each line will be numbered
	stringstyle=\color{mymauve},     % string literal style
	tabsize=3,	                     % sets default tabsize to 2 spaces
	title=\lstname                   % show the filename of files included with \lstinputlisting; also try caption instead of title	
}

%%% Ссылки
\usepackage{hyperref}

%%% Окружения
\theoremstyle{definition}
\newtheorem{theorem}{Теорема}
\newtheorem*{definition}{Определение}
\newtheorem*{pr}{Доказательство}

%%% Цифры в кружке
\newcommand*\circled[1]{\tikz[baseline=(char.base)]{
		\node[shape=circle,draw,inner sep=2pt] (char) {#1};}}

%%% Перечисление кастомное
\usepackage{enumitem}

%%% Естественный вывод
\usepackage{bussproofs}

%%% Несколько колонок для перечислений
\usepackage{multicol}

\usepackage{array}

%% no page numbering
\pagenumbering{gobble}

\begin{document}
	
	\subsection*{Задача 3}
	
	Построить машину Тьюринга, которая писала бы зеркальный образ заданного слова. 
	
	\subsection*{Решение}
	
	Алфавит для слов $\mathscr{U} = \{a_1, \dots, a_n\}$. Слово заданное на ленте --- 
	$a_i \: a_k \dots a_j$. Без удаления данного слова. 
	
	\begin{multicols}{4}
			
		\begin{tabular}{ | c |  }
			\hline
			Команды \\ 
			\hline
			$q_0 \; a_i \; m \; q_0^i$  \\
			\hline
			$q_0^i \; m \; R \; q_0^i$  \\
			$q_0^i \; a_l \; R \; q_0^i$  \\
			$q_0^i \; B \; R \; q_B^i$  \\
			\hline
			$q_B^i \; B \; a_i \; q_L^i$ \\
			$q_B^i \; a_l \; R \; q_B^i$ \\
			\hline
			$q_L^i \; a_l \; L \; q_L^i$ \\				
			$q_L^i \; B \; L \; q_L^i$  \\
			$q_L^i \; m \; a_i \; q_E$  \\
			\hline
			$q_E \; a_l \; L \; q_0$  \\
			\hline
		\end{tabular}
		
		\begin{tabular}{ | c | }
			\hline
			Конфигурации \\ 
			\hline
			$B \; (q_0 \; a_i) \; a_k \dots a_j \; B$  \\
			$B \; (q_0^i \; m) \; a_k \dots a_j \; B$   \\
			$B \; (q_0^i \; B) \; m \; a_k \dots a_j \; B$   \\
			$B \; (q_B^i \; B )\; B \; m \; a_k \dots a_j \; B$  \\
			$B \; (q_L^i \; a_i) \; B \; m \; a_k \dots a_j \; B$ \\
			$B  \; a_i \; (q_L^i \; B) \; m \; a_k \dots a_j \; B$  \\
			$B  \; a_i  \; B \; (q_L^i \; m) \; a_k \dots a_j \; B$  \\
			$B  \; a_i  \; B \; (q_E \; a_i) \; a_k \dots a_j \; B$   \\
			$B  \; a_i  \; B  \; a_i \; (q_0 \; a_k) \dots a_j \; B$  \\
			\dots \\
			$B  \; a_j \dots a_k \; a_i  \; B  \; a_i  \; a_k \dots a_j \; (q_0 \; B)$  \\
			\hline
		\end{tabular}
	\end{multicols}

	\subsubsection*{Таблица состояний и граф переходов}
	\begin{multicols}{3}
		\begin{itemize}
			\item[] 
				\begin{minipage}{\linewidth}\raggedright
					\begin{tabular}{ | c | c | c | c | }
						\hline
						& $a_l$ & $B$ & $m$ \\
						\hline
						$q_0$ & $q_0^l \; m$ & - & - \\
						\hline
						$q_0^i$ & $R \; q_0^i$ & $R \; q_B^i$ & $R \; q_0^i$ \\
						\hline
						$q_B^i$ & $R \; q_B^i$ & $q_L^i \; a_i$ & - \\
						\hline
						$q_L^i$ & $L \; q_L^i$ & $L \; q_L^i$ & $q_E \; a_i$ \\
						\hline
						$q_E$ & $L \; q_0$ & - & - \\
						\hline
					\end{tabular}
				\end{minipage}
			\item[] 
				\begin{minipage}{\linewidth}\raggedright
					\begin{tikzpicture}[node distance={25mm}, thick, main/.style = {draw, circle}] 
						\node[main] (0) {$q_0$}; 
						\node[main] (1) [right of=0] {$q_0^i$}; 
						\node[main] (2) [right of=1] {$q_B^i$}; 
						\node[main] (3) [right of=2] {$q_L^i$}; 
						\node[main] (4) [above right of=1] {$q_E$};
						
						\draw[->] (0) -- node[midway, below] {$a_l$} (1);
						
						\draw[->] (1) to [out=270,in=300,looseness=10] node[midway, right] {$a_l$} (1); 
						\draw[->] (1) to [out=255,in=285,looseness=10] node[midway, left] {$m$} (1);
						\draw[->] (1) -- node[midway, below] {$B$} (2); 
						
						
						\draw[->] (2) to [out=270,in=300,looseness=10] node[midway, right] {$a_l$} (2);
						\draw[->] (2) -- node[midway, below] {$B$} (3);
						
						\draw[->] (3) to [out=270,in=300,looseness=10] node[midway, right] {$a_l$} (3); 
						\draw[->] (3) to [out=255,in=285,looseness=10] node[midway, left] {$B$} (3);
						\draw[->] (3) -- node[midway, above] {$m$} (4);
						
						\draw[->] (4) -- node[midway, above] {$a_l$} (0);
					\end{tikzpicture}
				\end{minipage}
		\end{itemize}
	\end{multicols}	
	
\end{document}